\documentclass{article}

\usepackage{hyperref}
\usepackage{xcolor}
\usepackage{listings}
\usepackage{color}
\usepackage[utf8]{inputenc}

\usepackage{helvet}
\renewcommand{\familydefault}{\sfdefault}

\hypersetup{
	colorlinks,
	linkcolor={black!50!black},
	citecolor={blue!50!black},
	urlcolor={blue!80!black}
}

% Format code snippets
\lstset{
	basicstyle=\ttfamily,
	columns=fullflexible,
	frame=single,
	breaklines=true,
	postbreak=\mbox{\textcolor{red}{$\hookrightarrow$}\space},
}

% Figures and images related
\usepackage{graphicx}
\usepackage{float}

\usepackage{microtype}
\usepackage{fancyhdr}
\newcommand{\displayMetaInformation}{
	\pagestyle{fancy}
	\fancyhf{}
	\rfoot{\thepage}
	\rhead{\tiny 23. April, 2021}
	\lhead{Martin Hansen \\ \tiny Tlf: 31165870 \\ Adr.: Hyben Alle 56 1.MF, 2770}
}

\newcommand{\centeredHeadline}[1]{\begin{center}\large #1\end{center}}

\newcommand{\nextPage}[1]{\textit{(#1...)} \pagebreak}

\newcommand{\fig}[3]{\begin{figure}[H]
		\centering
		\includegraphics[width=#2]{#1}
		\caption{#3}
		\label{fig:#1}
\end{figure}}

\newcommand{\frontFig}[2]{\begin{figure}[H]
		\centering
		\includegraphics[width=#2]{#1}
		\label{fig:#1}
\end{figure}}

\setlength{\parindent}{0cm}

\author{Martin Hansen}
\title{Programming notes and snippets 2023}

\newcommand{\note}{\textbf{Note:} }

\begin{document}
	
	\frenchspacing
	\maketitle
	
	\pagebreak
	
	\centeredHeadline{Worth to know}
	
	This guide assumes the reader holds a fundemental knowledge of linux and bash shell.  
	
	\pagebreak
	
	\tableofcontents
	
	\pagebreak
	
	\section{SSH}
	\subsection{Generate a key pair}
	The simple command that uses \textit{rsa}:
	\begin{lstlisting}
		ssh-keygen
	\end{lstlisting}
	
	For manual selection of algorithm and bit size:
	
	\begin{lstlisting}
		ssh-keygen -t [rsa,dsa,ecdsa,ed25519,ed25519-sk] -b [size] -C "[email]" 
	\end{lstlisting}

	\note DSA and RSA is traditional algorithms previosly considered unbreakable  up until this quantum era. But in the near future
	with the rise of quantum computers, this may change due to the way quantum computers works. GitHub now recommends at least using \textit{ecdsa}. But for student and testing purposes, \textit{rsa} will do just fine.
	
	\subsection{What is vagrant?}
	
	Vagrant may be called "Docker on steorids" as the container is not an image, but a box which contains a fully fletched virtual machine with OS and "ready to use" software. It relies on virtualization which means you have to have a hypervisor like VmWare or VirtualBox installed. A Vagrant box is a container which holds the VM and like docker images, can be pulled or pushed to an registry. This is ideal for developers that wants to ensure same environment across different platforms; just like docker, but on a larger scale. Here we not only talking a simple service, but a whole environment of services. The advantage here seems to be the total lack of "it works on my computer, but not on  yours" concerns, but the disadvantage is slow startup times and heavy disk space consumption. Furthermore, a Vagrant Box needs to be bounded to vm software like Virtual Box or VmWare.
	
	\subsection{Fire up vagrant}
	
	There are two ways to (as to my understanding) you can initialize a project to use vagrant.
	
	\section{Docker}
	
	\subsection{What is docker?}
	Docker is a modern software tool within the category of software containerization. It takes many of the advantages of virtualization and leaves out disadvantages like slow start up time and heavy disk space consumption. Docker images relies on alpine versions of many modern operation systems which makes it ideal to containerize single services like small web servers and dbms.
	
	\subsection{Basic  operations}
	Show running containers:
	
	\begin{lstlisting}
		docker ps
	\end{lstlisting}
	
	List images:
	
	\begin{lstlisting}
		docker image ls
		docker images
		docker images -a
		docker images -q
	\end{lstlisting}

	\note{} The 'a' and 'q' arguments are used to list all images and only their ids, respectively.\\
	
	Remove all images:
	
	\begin{lstlisting}
		docker rm $(docker images -a -q)
	\end{lstlisting}

	Remove all containers:
	
	\begin{lstlisting}
		docker rm $(docker container ls -a -q)
	\end{lstlisting}

	\subsection{Official MySQL image}
	
	Retrieve image from dockerhub:
	
	\begin{lstlisting}
		docker pull mysql/mysql-server:{version [mostly use 'latest' or ommit value to use latest version]}
	\end{lstlisting}
	
	\begin{lstlisting}
		docker run -p {host port}:{container port} --name {container name}  -d mysql/mysql-server:tag
	\end{lstlisting}
	
	Enter mysql interface:
	\begin{lstlisting}
		docker exec -it {container name} mysql -uroot -p
	\end{lstlisting}

	\subsection{Alter root access hosts}
	For some scenarios you may encounter an issue, that prevents you from accessing the mysql interface from localhost.  Then you have to grant the root user access rights from any hosts (this is not really recommended, so don't do this in production or in any serious use case).\\
	\begin{lstlisting}
		CREATE USER 'root'@'%' IDENTIFIED BY 'password';
		GRANT ALL PRIVILEGES ON *.* TO 'root'@'%' WITH GRANT OPTION;
	\end{lstlisting}

\end{document}
